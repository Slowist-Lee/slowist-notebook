\documentclass[fontset=windows]{article}
\usepackage[zihao=-4]{ctex}
\usepackage[a4paper]{geometry}
\usepackage{graphicx}
\begin{document}
\title{Note of \LaTeX}  
\author{Slowist\_Lee}
\maketitle
\section{Introduction}
通篇文章可以分为三个部分:1我想写一篇文档 2.我要开始写一篇文档 3. 我正在写一篇文档。
\subsection{subsection}
副标题

\subsubsection{subsubsection}
副标题的副标题
\paragraph{Paragraph的作用} "paragraph"的作用只是另起一段并在开头添加了粗体的段标题,后面的文字将会紧跟标题,不会像section一样另起一段。同时可以看到左下角的outline:
\section{入门}


\paragraph{I.换行}方法是加空格之后直接换行【中间需要空行】 

	第一行第一行第一行第一行第一行第一行第一行第一行
	
第二行第二行第二行第二行第二行第二行第二行第二行
\paragraph{II.转义}
\{显示大括号\}

要注意\verb|\|这个符号的转义有点麻烦:
\begin{center}
\verb|\|
\end{center}
\paragraph{III.字体}


\textbf{黑体} 

 % 字体族设置:通过命令设置,{} 内为参数,即需要设置格式的文本
 \textrm{Roman Family} 


 \textsf{Sans Serif Family}


 \texttt{Typewriter Family} \par
 
 % 字体族设置:通过声明设置,声明后的文本全部按照该格式,{} 表示范围
 {\songti 宋体,通过声明设置} \par
 {\heiti 黑体,声明后的文本} \par
 {\fangsong 仿宋,按照该格式} \par
 {\kaishu 楷书,花括号表示范围} \par

    % 字体形状设置(直立,斜体,伪斜体 ,小型大写)
   % 字体命令
   \textup{Upright Shape 直立,}	
   \textit{Italic Shape 斜体,} \par
   \textsl{Slanted Shape 伪斜体,}
   \textsc{Small Caps Shape 小型大写} \par
   
   % 字体声明
   \upshape{直立 Upright Shape,}	
   \itshape{斜体 Italic Shape,}	\par
   \slshape{伪斜体 Slanted Shape,}	
   \scshape{小型大写 Small Caps Shape}

   \slshape{\texttt{ADDING:Typewriter Family} \par\par}	
   \upshape{恢复直立字体,裂开。}

   
   {\songti 
   
    subsection{1.4 字体大小的设置}
	
	字体大小在 \verb|\|documentclass\{article\} 中可以设置为 10pt, 11pt, 12pt。
	
	\verb|\|zihao \{〈字号〉\} 命令用于调整字号大小。其中 〈 字号 〉 的有效值共有 16 个,如表 所示。
	
	使用 \verb|\|zihao 命令调整字体大小时,英文字号大小会始终和中文字号保持一致。
 
% 字体的大小
{\tiny tiny: Hello!}\par
{\scriptsize scriptsize: Hello!}\par
{\footnotesize footnotesize: Hello!}\par
{\small small: Hello!}\par
{\normalsize normalsize: Hello!}\par
{\large large: Hello!}\par
{\Large Large: Hello!}\par
{\LARGE LARGE: Hello!}\par
{\huge huge: Hello!}\par
{\Huge Huge: Hello!}\par

% 中文的字号
\zihao{0} 你好!zihao(0)\par
\zihao{1} 你好!zihao(1)\par
\zihao{2} 你好!zihao(2)\par
\zihao{-2} 你好!zihao(-2)\par
\zihao{3} 你好!zihao(3)\par
\zihao{-3} 你好!zihao(-3)\par
\zihao{4} 你好!zihao(4)\par
\zihao{-4} 你好!zihao(-4)\par
\zihao{5} 你好!zihao(5)\par
\zihao{-5} 你好!zihao(-5)\par
\zihao{6} 你好!zihao(6)\par
\zihao{-6} 你好!zihao(-6)\par
\zihao{7} 你好!zihao(7)\par
\zihao{8} 你好!zihao(8)\par 

	} \par
\section{插入图片}


\end{document}


